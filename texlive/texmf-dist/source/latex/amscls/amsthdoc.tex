%% filename: amsthdoc.tex
%% version: 2.20.3
%% date: 2017/06/01
%% 
%% Copyright 1999, 2004, 2010, 2015, 2017 American Mathematical Society.
%%
%% American Mathematical Society
%% Technical Support
%% Publications Technical Group
%% 201 Charles Street
%% Providence, RI 02904
%% USA
%% tel: (401) 455-4080
%%      (800) 321-4267 (USA and Canada only)
%% fax: (401) 331-3842
%% email: tech-support@ams.org
%% 
%% This work may be distributed and/or modified under the
%% conditions of the LaTeX Project Public License, either version 1.3c
%% of this license or (at your option) any later version.
%% The latest version of this license is in
%%   http://www.latex-project.org/lppl.txt
%% and version 1.3c or later is part of all distributions of LaTeX
%% version 2005/12/01 or later.
%% 
%% This work has the LPPL maintenance status `maintained'.
%% 
%% The Current Maintainer of this work is the American Mathematical
%% Society.
%%
%% ====================================================================
%%
\documentclass[11pt,twoside]{article}
\pagestyle{myheadings}

% Use the same version number as for the public amsthm release.
\def\instrversion{2.20.3}
\def\instrdate{September 2017}

\title{Using the \pkg{amsthm} Package}
\author{Version \instrversion, \instrdate\\[.5ex]
 Publications Technical Group\\American Mathematical Society}
\date{}

\providecommand{\qq}[1]{\textquotedblleft#1\textquotedblright}
\providecommand{\mdash}{\textemdash\penalty\hyphenpenalty}

% Some item adjustments

\setlength\leftmargini{2em}
\setlength\leftmarginii{1.5em}
\renewcommand\labelitemii{$\circ$}

% Embedded \index commands are a legacy from the time when this
% documentation was part of amsldoc. Since they don't hurt anything,
% let's leave them in. Maybe they will become useful again in the
% future. [mjd,2000/06/06]

\chardef\bslchar=`\\ % p. 424, TeXbook
\newcommand{\ntt}{%
  \fontfamily\ttdefault \fontseries\mddefault \fontshape\updefault
  \selectfont
}
\DeclareRobustCommand{\cn}[1]{{\ntt\bslchar#1}}
\DeclareRobustCommand{\cls}[1]{{\ntt#1}}
\DeclareRobustCommand{\pkg}[1]{{\ntt#1}}
\DeclareRobustCommand{\opt}[1]{{\ntt#1}}
\DeclareRobustCommand{\env}[1]{{\ntt#1}}
\DeclareRobustCommand{\fn}[1]{{\ntt#1}}

\providecommand{\qedsymbol}{\leavevmode
  \hbox to.77778em{%
  \hfil\vrule
  \vbox to.675em{\hrule width.6em\vfil\hrule}%
  \vrule\hfil}}

%%  Provide a meta-command facility; provide an alternate escape
%%  character so it can be used within the verbatim environment.

\catcode`\|=0
\begingroup \catcode`\>=13 % in LaTeX2e verbatim env makes > active
\gdef\?#1>{{\normalfont$\langle$\textit{#1}$\rangle$}}
\gdef\0{\relax}
\endgroup
\def\<#1>{{\normalfont$\langle$\textit{#1}$\rangle$}}
\def\ntnote#1{{\normalfont$^{#1}$}\hangindent.5em}

\hfuzz4pt \vbadness9999 \hbadness5000
\def\latex/{{\protect\LaTeX}}

\setlength{\textwidth}{210mm}\addtolength{\textwidth}{-2in}
\setlength{\oddsidemargin}{39pt}
\setlength{\evensidemargin}{39pt}
\addtolength{\textwidth}{-2\oddsidemargin}

\setcounter{tocdepth}{2}

\usepackage{url}
\usepackage[breaklinks]{hyperref}

%%%%%%%%%%%%%%%%%%%%%%%%%%%%%%%%%%%%%%%%%%%%%%%%%%%%%%%%%%%%%%%%%%%%%%%%

\begin{document}
\maketitle
\markboth{Using the \pkg{amsthm} package}{Using the \pkg{amsthm} package}

\begingroup
\small
\tableofcontents
\endgroup

%%%%%%%%%%%%%%%%%%%%%%%%%%%%%%%%%%%%%%%%%%%%%%%%%%%%%%%%%%%%%%%%%%%%%%%%

\newpage %%%%%%%%%%%%%%%%%%%%

\section{Introduction}

\begin{sloppypar}
The \pkg{amsthm} package provides an enhanced version of \latex/'s
\cn{newtheorem} command for defining theorem-like environments. The
enhanced \cn{newtheorem} recognizes a
\cn{theoremstyle} specification (as in Mittelbach's \pkg{theorem}
package) and has a \verb+*+ form for defining unnumbered environments. The
\pkg{amsthm} package also defines a \env{proof} environment that
automatically adds a QED symbol at the end. AMS document classes
incorporate the \pkg{amsthm} package, so everything described here
applies to them as well.
\end{sloppypar}

As part of the AMS \latex/ support environment, \pkg{amsthm} follows
AMS style.  This differs in some respects from the style provided by
the base \latex/ classes and the package mentioned above; where a
difference is known to be present, it will be noted in these instructions.
One notable difference is that, after the end of a theorem or proof,
the following line of text is assumed to start a new paragraph, and will
always be indented, whether or not a blank line or \cn{par} is present.

If the \pkg{amsthm} package is used with a non-AMS document class and
with the \pkg{amsmath} package, \pkg{amsthm} must be loaded
\emph{after} \pkg{amsmath}, not before.

Since \pkg{amsthm} follows AMS publication style, formats that diverge
significantly from that style may not be able to be accommodated.
Other theorem packages \emph{do} exist, and users with incompatible
requirements who are publishing elsewhere are advised to look for other
possibilities on CTAN\@.  However, authors preparing files for
publication by the AMS are expected to follow AMS style.

This manual describes the features provided by \pkg{amsthm}.  Examples
are given separately in the file \fn{thmtest.tex}.  For best understanding,
examine the output side-by-side with the input.

%%%%%%%%%%%%%%%%%%%%%%%%%%%%%%%%%%%%%%%%%%%%%%%%%%%%%%%%%%%%%%%%%%%%%%%%

%\section{The \cn{newtheorem} command}
\section{Setup and input of theorem elements}

In mathematical research articles and books, theorems\index{theorems}
and proofs\index{proofs} are among the most common elements, but authors
also use many others that fall in the same general class: lemmas,
propositions, axioms, corollaries, conjectures, definitions, remarks,
cases, steps, and so forth (see page~\pageref{thmstyle:list}).
It is natural to handle these elements as
\latex/ environments, but document classes do not
provide predefined environments for theorem-like elements because
(a)~that would make it difficult for authors to exercise the necessary
control over the automatic numbering, and (b)~the variety of such
elements is so wide that it's not possible for a document class to
provide every one that will ever be needed. Instead there is a command
\cn{newtheorem}, similar in effect to \cn{newenvironment}, that makes it
easy for authors to set up the elements required for a particular
document.

The \cn{newtheorem} command has two mandatory arguments: the first one
is the environment name that the author would like to use for this
element; the second one is the heading text. For example,
\begin{verbatim}
  \newtheorem{lem}{Lemma}
\end{verbatim}
means that instances in the document of
\begin{verbatim}
  \begin{lem} Text text ... \end{lem}
\end{verbatim}
will produce
\[\makebox[.8\columnwidth]{%
  \textbf{Lemma 1.} \textit{Text text \dots}\hfill}\]
where the heading consists of the specified text \qq{Lemma} and an
automatically generated number and punctuation.  Note that there are
no fixed text assignments as there are with, e.g., \cn{chaptername}
(usually ``Chapter'' or ``Appendix'', depending on context); the heading
text in the output will be exactly what you specify in the input.

If \cn{newtheorem*} is used instead of \cn{newtheorem} in this
example, numbers will not be generated automatically for
any of the lemmas in the document. This form of the command can be
useful even if you have only one lemma and don't want it to be
numbered; more often, though, it is used to produce a special named
variant of one of the common theorem types. For example, if you have a
lemma that should get the heading \qq{Klein's Lemma}, then the statement
\begin{verbatim}
  \newtheorem*{KL}{Klein's Lemma}
\end{verbatim}
would allow you to write
\begin{verbatim}
  \begin{KL} Text text ... \end{KL}
\end{verbatim}
and get the desired output.

Sometimes additional information is desired in the heading of a
theorem or lemma, often because it is cited from another source.
An optional argument is used to provide this information:
\begin{verbatim}
  \begin{lem}[Alinhac-Lerner \cite{a-l}]
\end{verbatim}
yielding the output
\[\makebox[.8\columnwidth]{%
  \textbf{Lemma 2.} (Alinhac-Lerner [4]). \textit{Text text \dots}\hfill}\]
If the citation includes page numbers, the brackets in the reference
must be ``hidden'' to prevent the closing bracket from being interpreted
as the end of the optional argument to the heading:
\begin{verbatim}
  \begin{lem}[Alinhac-Lerner {\cite[p.~37]{a-l}}]
\end{verbatim}
with the output
\[\makebox[.8\columnwidth]{%
  \textbf{Lemma 2.} (Alinhac-Lerner [4, p.~37]). \textit{Text text \dots}\hfill}\]

Often, theorems are referred to elsewhere in a document.  The usual
\latex/ \cn{label}-\cn{ref} mechanism works for this purpose.  However,
problems can arise when the \cn{label} is separated from the heading.
For this reason, it is recommended to place the label immediately
after the \verb+\begin{...}[...]+ command; placing it on a separate line
should not cause any adverse effects.

\subsection{Theorems beginning with lists}
\label{ThmWithList}

Sometimes it's desired to state a theorem in discrete steps, using
a list.  Preceding the list by text is recommended, to avoid problems
both with linking (using \env{enumerate}) and with unattractive spacing.

If no text is provided, the list will follow the heading on the same
line.  This has three unwanted effects: the horizontal space between
the theorem head and the first item will be unacceptably large, a
hyperlink to the theorem will not be ``live'' in the PDF file
(although it will be marked), and the extra vertical space that should
follow the end of the list will be absent.

The best way to avoid these problems is to allow the list to start
on a new line.  One way to accomplish this is to follow the theorem
head (and \cn{label}, if present) by the command \cn{leavevmode}.
(For more information, see section~\ref{thmstyle:break}.) 
However, if the theorem comes near the bottom of a page, the list
might move to the top of the next page, leaving an orphaned heading.
If this happens, it must be addressed as an exception, to be taken
care of after the text is final; at that point, the recommended
fix is to call on the \pkg{needspace} \cite{NDS} package, or insert
an explicit page break.

An alternative method of starting a new line after the heading is to
provide a \verb+\newtheoremstyle{break}+; the definition is given below
on page~\pageref{thmstyle:break}.  Like the \cn{leavevmode} approach,
a \env{break} theorem environment is not perfect; known limitations
accompany the definition.

%The recommended remedy is to insert the command \cn{leavevmode}
%after the heading (and label, if present).  This will move the first
%item to a new line, and restore linking and the usual spacing.
%Adding \cn{samepage} immediately following \cn{leavevmode} will
%prevent the page break, but this has the side effect of treating the
%entire theorem as a unit, so such an adjustment should be delayed until
%the ``last pass'' to minimize possible problems with long theorems.

%% see also "break" below

%% reconsider the following
%% to be explored: use of enumerate package to reset item id styles

If the list \emph{must} start on the same line as the heading, enter
the first item identifier manually, with adjustments: enter an
ordinary space between the heading and the identifier, reset the left
margin if the item text will require more than one line, and reset
the item counter (for \env{enumerate}) so that it will start
with 2 (or whatever is appropriate).
\begin{verbatim}
  \begin{thm}[|?optional modifier>]
  \hangindent\leftmargini  % for a multi-line item
  \label{thm:xxx}
  \textup{(1)} First item ...
  \begin{enumerate}
  \setcounter{enumi}{1}
  \item ...
  \item ...
  \end{enumerate}
  \end{thm}
\end{verbatim}
The style shown here for the item identifier matches the default for
AMS document classes.  It may be different for other document classes;
\pkg{amsthm} retains the class default.  (\cn{textup} should be
applied to hand-coded item identifiers when using an AMS class; AMS
style conforms to traditional math typography in this respect, with item
identifiers, cross-references, and similar elements consistently set
upright, even in italic environments.  However, an author should always
try to match the style of the document class being used.)  To match the
default style of a first-level item counter when using an AMS document
class, a reference may be entered with \cn{eqref}; however, beware if
a package such as \pkg{cleveref} is being used, as it may associate the
wrong element type.

An initial item input in this manner cannot be linked using a \cn{label},
since it is not associated with a counter; the \cn{label} in the above
example refers to the theorem element, not to the item in the list.
This is an important concept: a \cn{label} will always refer to the
value of the most recent counter that is automatically incremented;
this counter will also be the anchor for a hyperlink if the
\pkg{hyperref} package is used.

%%% theorems starting with lists; linking -- xref

%%% footnote on header -- Q-586

%%%%%%%%%%%%%%%%%%%%%%%%%%%%%%%%%%%%%%%%%%%%%%%%%%%%%%%%%%%%%%%%%%%%%%%%

\enlargethispage{1\baselineskip} %%%%%%%%%%%%%%%%%%%%

\section{Theorem numbering}

In addition to the two mandatory arguments, \cn{newtheorem} has two
mutually exclusive optional arguments. These govern the
sequencing\index{theorems!numbering} and hierarchy of the numbering.
The numbering mechanism can be thought of this way:
%\null\hfil\verb+\newtheorem{+\<environment name>\verb+}[+\<shared counter>%
\begin{verbatim}
  \newtheorem{|?env name>}{|?text>}[|?parent counter>]
  \newtheorem{|?env name>}[|?shared counter>]{|?text>}
\end{verbatim}
%(Only one of these optional arguments can be present at one time.)
The \<parent counter> is comparable to \cn{numberwithin}; that is,
numbering will restart whenever that sectional level is encountered.
If a \<shared counter> is specified, numbering will progress
sequentially for all theorem elements using this counter.  See
below for a more detailed explanation.

%% ref tex.sx q/5244 for clear explanation

By default each kind of theorem-like environment is numbered
independently. Thus if you have three lemmas and two theorems
interspersed, they will be numbered something like this: Lemma 1, Lemma
2, Theorem 1, Lemma 3, Theorem 2. If you want lemmas and theorems to
share the same numbering sequence\mdash Lemma 1, Lemma 2, Theorem 3, Lemma
4, Theorem 5\mdash then you should use the \<shared counter> to indicate
the desired relationship as follows:
\begin{verbatim}
  \newtheorem{thm}{Theorem}
  \newtheorem{lem}[thm]{Lemma}
\end{verbatim}
The optional argument \verb+[thm]+ in the second statement means that
the \env{lem} environment should share the \env{thm} numbering
sequence instead of having its own independent sequence.

To have a theorem environment numbered subordinately within a
sectional unit\mdash e.g., to get propositions numbered Proposition 2.1,
Proposition 2.2, and so on in Section~2\mdash put the name of the parent
unit in square brackets in final position:
\begin{verbatim}
  \newtheorem{prop}{Proposition}[section]
\end{verbatim}
With the optional argument \verb+[section]+, the \verb+prop+ counter
will be reset to 0 whenever the parent counter \verb+section+ is
incremented, and the proposition heading will have the section number
prepended.

If any theorem elements are numbered by section, and (in a book) the
first such element in a chapter comes before the first section,
numbering will continue from the previous chapter.  In such a case,
reset the counter by invoking this command before the affected element:
\begin{verbatim}
  \setcounter{thm}{0}
\end{verbatim}

Note that theorem numbering is not accomplished by the same method
as the \cn{numberwithin} facility of \pkg{amsmath}, so an attempt to
use that to relate theorem numbers to equation numbers will not work
in the expected way.  A method for accomplishing that, as well as
some other variations, is given in the AMS Author FAQ \cite{AF},
in the section on
\qq{\href{http://www.ams.org//faq?class_id=561}{Theorems in AMS-\LaTeX}}.

\newpage %%%%%%%%%%%%%%%%%%%%

\subsection{Related topics addressed in the AMS author FAQ}

\begin{itemize}
\item \href{http://www.ams.org/faq?faq_id=199}{``Number''
  theorems with letters instead of numbers}

\item \href{http://www.ams.org/faq?faq_id=200}{Restart numbering for each
  chapter, but don't include chapter number as part of theorem number}
 
\item \href{http://www.ams.org/faq?faq_id=202}{Number theorems by
  section when chapters have sections, but by chapter only when a chapter
  has no sections}

\item \href{http://www.ams.org/faq?faq_id=289}{Number theorems and
  equations with the same numbering sequence}
\end{itemize}

%%%%%%%%%%%%%%%%%%%%%%%%%%%%%%%%%%%%%%%%%%%%%%%%%%%%%%%%%%%%%%%%%%%%%%%%

\section{Changing styles for theorem-like environments}

\subsection{The \texorpdfstring{\cn{theoremstyle}}{theoremstyle} command}

The \pkg{amsthm} package supports the notion of a current theorem style,
which determines what will be produced by a given \cn{newtheorem}
command. The three theorem styles provided\mdash
\env{plain},\index{plain theo@\env{plain} theorem style}\relax
\index{definition theo@\env{definition} theorem style}\relax
\index{remark theo@\env{remark} theorem style} \env{definition},
and \env{remark}\mdash specify different degrees of
visual emphasis corresponding to their relative importance. The
details of this typographical treatment may vary depending on the
document class, but typically the \env{plain} style produces italic
body text, while the other two styles produce roman body text.  These
default settings are provided:
\begin{itemize}
\item \env{plain}\,: italic text, extra space above and below;
\item \env{definition}\,: upright text, extra space above and below;
\item \env{remark}\,: upright text, no extra space above or below.
\end{itemize}

If no \cn{theoremstyle} command is given, the style used will be
\env{plain}. To specify different styles, divide your \cn{newtheorem}
commands into groups and preface each group with the appropriate
\cn{theoremstyle}. Some examples:\label{thmstyle:list}
\begin{verbatim}
  \theoremstyle{plain}% default
  \newtheorem{thm}{Theorem}[section]
  \newtheorem{lem}[thm]{Lemma}
  \newtheorem{prop}[thm]{Proposition}
  \newtheorem*{cor}{Corollary}
  \newtheorem*{KL}{Klein's Lemma}

  \theoremstyle{definition}
  \newtheorem{defn}{Definition}[section]
  \newtheorem{exmp}{Example}[section]
  \newtheorem{xca}[exmp]{Exercise}

  \theoremstyle{remark}
  \newtheorem*{rem}{Remark}
  \newtheorem*{note}{Note}
  \newtheorem{case}{Case}
\end{verbatim}

The following list summarizes the types of structures which are
normally associated with each theorem style.

\smallskip
\begin{tabular}{ll}
 \relax\env{plain} & Theorem, Lemma, Corollary, Proposition, Conjecture, \\
                   & Criterion, Assertion \\
 \relax\env{definition} \quad & Definition, Condition, Problem, Example, 
		     Exercise, \\
		   & Algorithm, Question, Axiom, Property, Assumption, \\
		   & Hypothesis \\
 \relax\env{remark} & Remark, Note, Notation, Claim, Summary, \\
                    & Acknowledgment, Case, Conclusion
\end{tabular}

\subsection{Number swapping}

A common style variation is to have the theorem number at the beginning
of the heading instead of at the end, for example \qq{1.4 Theorem}
instead of \qq{Theorem 1.4}.  As this variation is usually applied
to all \cn{theoremstyle}s, it is done by
placing a \cn{swapnumbers} command at the beginning of the list of
\cn{newtheorem} statements that should be affected. For example:
\begin{verbatim}
  \swapnumbers
  \theoremstyle{plain}
  \newtheorem{thm}{Theorem}[section]
  \theoremstyle{remark}
  \newtheorem{rem}{Remark}
\end{verbatim}

When \pkg{amsthm} is used with a non-AMS document class, and numbers are
swapped, no period is set following the number; this differs from the
``basic'' \latex/ style.  A workaround will be given in the AMS Author
FAQ~\cite{AF}.

In some AMS document classes, the style of the swapped numbers matches
that of section headings; this may not be the same style as the rest
of the theorem heading.

% amsart: lightface numbers, bold heading
% amsproc: all lightface; section headings bold
% amsbook: all lightface; section headings bold

\newpage %%%%%%%%%%%%%%%%%%%%

\enlargethispage{1\baselineskip} % to fit code on same page
\subsection{New theorem styles}

The \pkg{amsthm} package provides a \cn{newtheoremstyle} command
to aid in the creation of custom styles, should the three predefined
styles prove insufficient.  The following example illustrates the use
of the \cn{newtheoremstyle} command.
\begin{verbatim}
  \newtheoremstyle{note}%  |?name>
    {3pt}%      |?Space above>|ntnote1
    {3pt}%      |?Space below>|ntnote1
    {}%         |?Body font>
    {}%         |?Indent amount>|ntnote2
    {\itshape}% |?Theorem head font>
    {:}%        |?Punctuation after theorem head>
    {.5em}%     |?Space after theorem head>|ntnote3
    {}%         |?Theorem head spec |textup(can be left empty, meaning `normal'|textup)>
\end{verbatim}
\noindent
\ntnote1 Space above and below: these are equal, but the defaults differ
 when using an AMS document class vs.\ \verb+\usepackage{amsthm}+. For
 AMS classes, this is \verb+.5\baselineskip+$\pm$\verb+.2\baselineskip+;
 for \pkg{amsthm} used as a separate package, it is the local value
 of \cn{topsep}.  If these arguments are left empty, the current
 defaults are used.

\noindent
\ntnote2 Indent amount: empty = no indent, \cn{parindent} = normal
 paragraph indent

\noindent
\ntnote3 Space after theorem head: \verb+{ }+ = normal interword space; \\
%\null\phantom{\ntnote2}
 \cn{newline} = line break

\smallskip
Apply the new theorem style in the same manner as the predefined ones.

Here are some styles that have been requested by authors.

\subsubsection{Theorem style \texttt{break}}
\label{thmstyle:break}
This style will break after the theorem heading and start a new line.
\begin{verbatim}
\newtheoremstyle{break}%
  {}{}%
  {\itshape}{}%
  {\bfseries}{}%  % Note that final punctuation is omitted.
  {\newline}{}
\end{verbatim}

This style can be used for a theorem beginning with a list.
When used with \env{enumerate} and an AMS document class, all items
are properly labeled and will link.  However, the vertical spacing
needs help; a conflict between definitions prevents the first
item from starting on a new line---it looks almost the same as
a default theorem beginning with an enumerated list.
To repair this problem, begin the theorem like this:
\begin{verbatim}
  \begin{breakthm}[...]
  \leavevmode \vspace{-\baselineskip}
  \begin{enumerate}
\end{verbatim}
\verb+\leavevmode+ alone will leave a full blank line after the theorem head.

One more problem may arise: if the theorem starts close to the end
of a page, the list could be split to a new page, leaving an
orphaned heading.  Make a note to address this when the text is
final; then call on the \pkg{needspace} \cite{NDS} package, or insert
an explicit page break.

\subsubsection{Theorem with bold note field}

If the note---with surrounding parentheses---is particularly important,
setting the note field in bold may be desired.  This variation numbers
the theorem, an option not available when using \cn{newtheorem*}.
\begin{verbatim}
\newtheoremstyle{bfnote}%
  {}{}%
  {\itshape}{}%
  {\bfseries}{.}%
  { }%
  {\thmname{#1}\thmnumber{ #2}\thmnote{ (#3)}}
\end{verbatim}

Parentheses may be omitted from around the note using this trivial
adaptation of the formulation with a bold note.
\begin{verbatim}
\newtheoremstyle{noparens}%
  {}{}%
  {\itshape}{}%
  {\bfseries}{.}%
  { }%
  {\thmname{#1}\thmnumber{ #2}\thmnote{ #3}}
\end{verbatim}

%\subsubsection*{other suggested examples (to be considered)}
%
%\begin{verbatim}
% Q-353 [\newtheoremstyle, used with numbers omits parens on text]
% Q-426 [\newtheoremstyle, number font doesn't change with \@thmheadfont]
%\end{verbatim}

\subsubsection{Other styles}
For examples, see the file \fn{thmtest.tex}.

%%%%%%%%%%%%%%%%%%%%%%%%%%%%%%%%%%%%%%%%%%%%%%%%%%%%%%%%%%%%%%%%%%%%%%%%

\newpage %%%%%%%%%%%%%%%%%%%%

\section{Proofs}

A predefined \env{proof} environment provided by the \pkg{amsthm}
package produces the heading \qq{Proof} with appropriate spacing and
punctuation. The proof environment is primarily intended for short
proofs, no more than a page or two in length; longer proofs are
usually better done as a separate \cn{section} or \cn{subsection} in
your document.

\subsection{The beginning of a proof}

An optional argument of the proof environment allows you to substitute
a different name for the standard \qq{Proof\/}. If you want
the proof heading to be, say, \qq{Proof of the Main Theorem}, then write
\begin{verbatim}
  \begin{proof}[Proof of the Main Theorem]
\end{verbatim}

Because of the way the \env{proof} environment is defined (as a
\latex/ list), a substitute ``name'' that is longer than one output
line will not break as it should; this is on the list for repair.
In the meantime, this redefinition in your preamble will accomplish
the desired result.
\begin{verbatim}
  \makeatletter
  \renewenvironment{proof}[1][\proofname]{\par
    \pushQED{\qed}%
    \normalfont \topsep6\p@\@plus6\p@\relax
    \trivlist
    \item\relax
          {\itshape
      #1\@addpunct{.}}\hspace\labelsep\ignorespaces
  }{%
    \popQED\endtrivlist\@endpefalse
  }
  \makeatother
\end{verbatim}

A proof beginning with a list can be approached in the same manner
as a theorem; see section~\ref{ThmWithList}.

\subsection{The end of a proof}

A QED symbol, \qedsymbol, is automatically appended at the end of
a \env{proof} environment. To substitute a different end-of-proof
symbol, use \cn{renewcommand} to redefine the command \cn{qedsymbol}.
For a long proof done as a subsection or section instead of with the
\env{proof} environment, you can obtain the symbol and the usual
amount of preceding space by using \cn{qed}; this will work properly
only if it follows ordinary text.

A blank line preceding \verb+\end{proof}+ (or before an explicit \cn{qed})
disables the mechanism that places the QED symbol, and may even allow
that symbol to be set at the top of a new page.

Placement of the QED symbol can be problematic if the last part of
a \env{proof} environment is a displayed equation or list environment
or something of that nature. In that case put a \cn{qedhere} command at
the place where the QED symbol should appear, for example,
\begin{verbatim}
  \begin{proof}
  ...
  \begin{equation}
  G(t)=L\gamma!\,t^{-\gamma}+t^{-\delta}\eta(t) \qedhere
  \end{equation}
  \end{proof}
\end{verbatim}
When used with the \pkg{amsmath} package, version 2 or later, \cn{qedhere}
will position the QED symbol flush right; with earlier versions, the
symbol will be spaced a quad away from the end of the text or display.

If \cn{qedhere} produces an error message in an equation, try using \\
\verb+\mbox{\qedhere}+ instead.

\cn{qedhere} does not work with \env{eqnarray} or with equations input
with \verb+$$+; the display environments defined by \pkg{amsmath} and
the \verb+\[ ... \]+ form of unnumbered equations have been specially
crafted for this purpose.

When a proof ends with a list environment, place \cn{qedhere} as the
last thing before \verb+\end{enumerate}+ or \verb+\end{itemize}+.

If, for some reason, it is desired to omit the QED symbol from the
end of a proof, it can be suppressed by including this definition just
before \verb+\end{proof}+:
\begin{verbatim}
  \renewcommand{\qedsymbol}{}
\end{verbatim}
If the symbol is to be omitted from \emph{all} proofs in the document,
place the redefinition in the preamble.

If the QED symbol is needed outside of the \env{proof} environment,
use \cn{qed} rather than \cn{qedhere}.

%%%%%%%%%%%%%%%%%%%%%%%%%%%%%%%%%%%%%%%%%%%%%%%%%%%%%%%%%%%%%%%%%%%%%%%%

\newpage %%%%%%%%%%%%%%%%%%%%

\section{Known problems}

The following problems are listed for attention with the next update
of \pkg{amsthm}.

\begin{itemize}
\item Problems with \cn{qedhere}. % B-366
 \begin{itemize}
  \item When \cn{qedhere} is used with an unstarred multi-line display
    structure, the final line is not numbered.  (This is relevant only
    when numbers are on the left; \cn{qedhere} is inappropriate for
    use with displays numbered on the right.)
  \item When \cn{qedhere} is within a \env{split} environment, that
    part of the display is not numbered and \cn{label} is suppressed.
    If \cn{qedhere} is moved outside of \env{split} the number appears
    but the content of the \env{split} is shifted left, and the QED
    symbol may be absent.
  \item In an alignment, if no \verb+&+ is present, the box does not
    go flush to the right margin.
  \item For a one-line equation, both the number and the box appear,
    but the equation is shifted left.
 \end{itemize}
\item Other problems inherent in \pkg{amsthm}.
 \begin{itemize}
  \item Hyperlinking the first item when a theorem begins with a list.
 \end{itemize}
\item Problems of interaction with AMS document classes.  Corrections
  will be made in the classes, not in \pkg{amsthm}.
 \begin{itemize}
  \item In \cls{amsbook}, when a theorem appears before a chapter's
    first numbered section, theorem numbering continues from the
    previous chapter.
 \end{itemize}
\end{itemize}

%%%%%%%%%%%%%%%%%%%%%%%%%%%%%%%%%%%%%%%%%%%%%%%%%%%%%%%%%%%%%%%%%%%%%%%%

\section{Other packages}

Some packages provide additional theorem functionality, with options
that recognize the presence of \pkg{amsthm} and adjust for it.
\begin{itemize}
\itemsep=.5\itemsep
\item \pkg{mdframed} \cite{MDF} provides the ability to set off theorems
  (and other elements) with frames of various shapes and colors.
  It should be loaded \emph{after} \pkg{amsthm} if both are used.
\item \pkg{thmtools} \cite{THT} provides an alternate key-value syntax
  for theorem specification, as well as variant presentation styles,
  both predefined and user definable.  A QED symbol can be applied to any
  theorem element, not just to proofs.  An adaptable \cn{listoftheorems}
  is also available.
\end{itemize}
For details, refer to the package manuals.  Both packages are included
in \TeX~Live, so can be accessed with
\texttt{texdoc}~\<package>.

Some packages that offer potentially useful features have limitations
that conflict with the implementation of theorems with AMS document
classes or \pkg{amsthm}.  Here are some we know about.
\begin{itemize}
\itemsep=.5\itemsep
\item \pkg{paralist} overrides the settings in \pkg{amsthm} that apply
  \cn{upshape} to item numbers in an \env{enumerate} environment within
  a theorem; the numbers become italic, which is undesired.
\item \pkg{wrapfig} documents that it does not coexist with lists.
  Both theorems and proofs in \pkg{amsthm} are based on \cn{trivlist}.
  An insertion using this method may cause succeeding paragraphs to be
  indented by the width of the insert even if the depth of the
  insertion does not warrant it.
\item \pkg{floatflt} and \pkg{picinpar} suffer from the same limitation.
\end{itemize}

%%%%%%%%%%%%%%%%%%%%%%%%%%%%%%%%%%%%%%%%%%%%%%%%%%%%%%%%%%%%%%%%%%%%%%%%

\section{Potential modifications}

Feedback from users and experience with AMS publications have pointed
out several areas where added flexibility can benefit both users of
\pkg{amsthm.sty} and developers of ``derivative'' document classes
based on the AMS set.

These suggestions are relatively easy to implement:

\begin{itemize}
\item Change hard-coded font specifications (e.g., \cn{itshape})
  to commands (e.g., \cn{proofheadingfont}) to allow modification.
\item Provide commands instead of hard-coded values for other elements,
  such as punctuation.
\end{itemize}

The following suggestions, while having merit, require more work, including
redesign of existing code:

\begin{itemize}
\item Extend applicability of \cn{qedhere} to theorem-class elements
  as well as to proofs.
\item Provide a means to ignore the definition of \env{proof} to
  avoid name conflicts with some non-AMS document classes.
\item Consider key-value mechanism for specifying \cn{newtheoremstyle}
  options.
\item Provide a new theorem style that begins with a list.
\end{itemize}

%%%%%%%%%%%%%%%%%%%%%%%%%%%%%%%%%%%%%%%%%%%%%%%%%%%%%%%%%%%%%%%%%%%%%%%%

\section{Providing feedback}

Reports on bugs found in this package should be submitted to\\
 \null\hfil \href{mailto:tech-support@ams.org}{\texttt{tech-support@ams.org}}

\noindent
For best results, a brief, compilable example should accompany the report.

An alternate method is to use the
  \href{http://latex-project.org/bugs-upload.html}{\LaTeX\ bugs}
  reporting mechanism, specifying \texttt{AMS-LaTeX} as the bug category.

%%%%%%%%%%%%%%%%%%%%%%%%%%%%%%%%%%%%%%%%%%%%%%%%%%%%%%%%%%%%%%%%%%%%%%%%

\begin{thebibliography}{[MDF]}

\raggedright

\bibitem[AF]{AF} AMS Author FAQ,
  \url{http://www.ams.org/authors/author-faq}

\bibitem[MDF]{MDF} The \pkg{mdframed} package,
  Marco Daniel and Elke~Schubert, 2013/07/01, v1.9b,
%  \url{http://mirror.ctan.org/macros/latex/contrib/mdframed/mdframed.pdf}
  \url{http://mirror.ctan.org/macros/latex/contrib/mdframed}

\bibitem[NDS]{NDS} The \pkg{needspace} package,
  Peter Wilson, 2010/09/12, v1.3d,
  \url{http://mirror.ctan.org/macros/latex/contrib/needspace}

\bibitem[THT]{THT} \pkg{Thmtools} Users' Guide,
  Ulrich M. Schwarz, 2014/04/21 v66,
%  \url{http://mirror.ctan.org/macros/latex/exptl/thmtools/thmtools.pdf}
  \url{http://mirror.ctan.org/macros/latex/exptl/thmtools}

\end{thebibliography}

\end{document}
